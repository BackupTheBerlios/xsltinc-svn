\documentclass{beamer}

\mode<presentation>
{
  \setbeamertemplate{background canvas}[vertical shading][bottom=red!10,top=blue!10]

  \usetheme{Warsaw}
  \usefonttheme[onlysmall]{structurebold}
}

\usepackage[french]{babel}
\usepackage[latin1]{inputenc}
\usepackage{times}
\usepackage[T1]{fontenc}

\title 
{Bureau Num�rique : Transformations Incr�mentales}

\subtitle
{Phase 2 : R�alisation}

\author
{Mika�l~Barbero \and C�dric~Brun}

\institute
{
	�cole Polytechnique de l'universit� de Nantes\\
	D�partement Informatique}

\date
{Projet de recherche 2005, 2006\\
 Pr�sent� le 18 janvier 2006}

\begin{document}

\begin{frame}
  \titlepage
\end{frame}

\begin{frame}
  \frametitle{Plan de la pr�sentation}
  \tableofcontents[pausesections]
\end{frame}

\AtBeginSubsection[]
{
  \begin{frame}<beamer>
    \frametitle{Plan de la pr�sentation}
    \tableofcontents[currentsection]
  \end{frame}
}

\section{Motivations et objectifs}

  \begin{frame}
    \frametitle{Rappel de la probl�matique}
    \framesubtitle{Le traitement des documents num�riques}
    \begin{itemize}
    \item
			tata
    \item
			tutu
    \item
			titi
    \end{itemize}
  \end{frame}
  
  \begin{frame}
    \frametitle{Rappel de la probl�matique}
    \framesubtitle{Le traitement des documents XML}
    \begin{block}{Besoin}<1->
      Am�liorer l'exp�rience utilisateur.
    \end{block}

    \begin{alertblock}{Probl�me}<2->
      Comment permettre l'�dition des documents r�sultant d'une transformation ?
    \end{alertblock}

    \begin{exampleblock}{Solution}<3->
      Rendre les tranformations inversibles.
    \end{exampleblock}  
  \end{frame}  
  
\section{Travail effectu�}

\section{D�monstration}

\section{Perspectives}

\section{Conclusion}

\section{Annexe : technologies utilis�es}

	\subsection{Introspection et r�flexion}
	
	\subsection{Programmation orient�e aspect}

\end{document}
