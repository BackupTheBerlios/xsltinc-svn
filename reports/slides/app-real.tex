\documentclass{beamer}

\mode<presentation>
{
  \setbeamertemplate{background canvas}[vertical shading][bottom=red!10,top=blue!10]

  \usetheme{Warsaw}
  \usefonttheme[onlysmall]{structurebold}
}

\usepackage[french]{babel}
\usepackage[latin1]{inputenc}
\usepackage{times}
\usepackage[T1]{fontenc}

\title 
{Bureau Num�rique : Transformations Incr�mentales}

\subtitle
{Annexes}

\author
{Mika�l~Barbero \and C�dric~Brun}

\institute
{
	�cole Polytechnique de l'universit� de Nantes\\
	D�partement Informatique}

\date
{Projet de recherche 2005, 2006\\
 Pr�sent� le 18 janvier 2006}

\begin{document}

\begin{frame}
  \titlepage
\end{frame}

\begin{frame}
  \frametitle{Plan de la pr�sentation}
  \tableofcontents[pausesections]
\end{frame}

\AtBeginSection[*]
{
  \begin{frame}<beamer>
    \frametitle{Plan de la pr�sentation}
    \tableofcontents[currentsection]
  \end{frame}
}

\section{Introspection et r�flexion}

  \begin{frame}
    \frametitle{Introspection}
    \framesubtitle{D�finition}
    \begin{itemize}
	    \item
    \end{itemize}
  \end{frame}

  \begin{frame}
    \frametitle{Introspection}
    \framesubtitle{�xemple}
    \begin{itemize}
	    \item
    \end{itemize}
  \end{frame}
  
  \begin{frame}
    \frametitle{Introspection}
    \framesubtitle{�xemple en Python}
    \begin{itemize}
	    \item
    \end{itemize}
  \end{frame}  

  \begin{frame}
    \frametitle{R�flexion}
    \framesubtitle{D�finition}
    \begin{itemize}
	    \item
    \end{itemize}
  \end{frame}

  \begin{frame}
    \frametitle{R�flexion}
    \framesubtitle{�xemple}
    \begin{itemize}
	    \item
    \end{itemize}
  \end{frame}
  
  \begin{frame}
    \frametitle{R�flexion}
    \framesubtitle{�xemple en Python}
    \begin{itemize}
	    \item
    \end{itemize}
  \end{frame}  
  
\section{Programmation orient�e aspect}
	
  \begin{frame}
    \frametitle{Programmation Orient�e Aspect (AOP)}
    \framesubtitle{Historique}
    \begin{itemize}
	    \item
    \end{itemize}
  \end{frame}

  \begin{frame}
    \frametitle{Programmation Orient�e Aspect (AOP)}
    \framesubtitle{Principes}
    \begin{itemize}
	    \item
    \end{itemize}
  \end{frame}
  
  \begin{frame}
    \frametitle{Programmation Orient�e Aspect (AOP)}
    \framesubtitle{Le m�thodes $after()$, $before()$ et $around()$}
    \begin{itemize}
	    \item
    \end{itemize}
  \end{frame}  
  
  \begin{frame}
    \frametitle{Programmation Orient�e Aspect (AOP)}
    \framesubtitle{Un exemple en Python}
    \begin{itemize}
	    \item
    \end{itemize}
  \end{frame}  
  
\end{document}
