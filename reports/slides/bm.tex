\documentclass{beamer}



\mode<article> % only for the article version
{
  \usepackage{fullpage}
  \usepackage{hyperref}
}

\mode<presentation>
{
  \setbeamercovered{transparent}

  \usetheme{Warsaw}
  \usefonttheme[onlysmall]{structurebold}
}

\usepackage[french]{babel}
\usepackage[latin1]{inputenc}
\usepackage{times}
\usepackage[T1]{fontenc}

\title 
{Bureau Num�rique : Transformations Incr�mentales}

\subtitle
{Phase 1 : Bibliographie et Mod�lisation}

\author
{Mika�l~Barbero \and C�dric~Brun}

\institute
{
	�cole Polytechnique de l'universit� de Nantes\\
	D�partement Informatique}

\date
{Projet de recherche 2005, 2006\\
 Pr�sent� le 15 novembre 2005}

\begin{document}

\begin{frame}
  \titlepage
\end{frame}

\begin{frame}
  \frametitle{Plan de la pr�sentation}
  \tableofcontents[pausesections]
\end{frame}

\AtBeginSection[]
{
  \begin{frame}<beamer>
    \frametitle{Plan de la pr�sentation}
    \tableofcontents[currentsection]
  \end{frame}
}

%% C�dric : Intro du plan et on se presente 

%% Mika
\section{Motivations et objectifs}

  \begin{frame}
    \frametitle{Cadre du Travail}
    \framesubtitle{Le Bureau Num�rique}
    \begin{itemize}
    \item
      Nouvelle vision du document.
    \item
      Augmentation de la taille des documents.
    \item
      Meilleure accessibilit� pour l'utilisateur.
    \end{itemize}
  \end{frame}

  \begin{frame}
    \frametitle{Probl�matique}
    \framesubtitle{Le Bureau Num�rique}
    \begin{itemize}
    \item
      Probl�me : traiter des documents de plus en plus importants 
      \begin{itemize}
        \item
          Reponse : traitement incr�mental
      \end{itemize}
    \item
      Probl�me : am�liorer l'exp�rience utilisateur
      \begin{itemize}
        \item
          R�ponse : transformation inverse
      \end{itemize}
    \end{itemize}
  \end{frame}
  
\section{�tat de l'art}
  %% C�dric
  \subsection{Notion de document}
  %% Mika (XML), C�dric (XPath) puis Mika (XSLT)
  \subsection{XML et XSL}
  %% C�dric
  \subsection{Traitements Incr�mentaux}
  %% Mika
  \subsection{Transformations Inverses}

\section{Conception}
  %% C�dric
  \subsection{Choix de Plate-Forme de D�veloppement}
  %% C�dric
  \subsection{4Suite et 4XSLT}
  %% Mika
  \subsection{Mod�lisation}
  
%% C�dric  
\section{Gestion de projet}
  \begin{frame}
    \frametitle{Mod�le de D�veloppement}
  \end{frame}
  \begin{frame}
    \frametitle{Plannification}
  \end{frame}

%% Mika
\section{Conclusion}
  \begin{frame}
    \frametitle{Conclusion}
  \end{frame}

\end{document}